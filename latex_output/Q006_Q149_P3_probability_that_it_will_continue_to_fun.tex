Sure, I can help you understand the question from this image using Python's OpenCV library to extract textual information (using pytesseract for OCR) along with some basic mathematical operations in latex formatting as follows: 

\textbf{Question} : What is a^2 + b^3 - c?  
The question asks us to find the value of 'c' given values from variables $a$ and $b$  which are not provided. The problem type can be identified by looking at pytesseract output for textual information related with this particular equation (which is a cubic sum). 
\begin{enumerate}
    \item Identify the types of problems: This question involves mathematical operations, so it's more suited to statistics or algebra.  
    $$a^2 + b^3 - c = 0 $$ which can be rewritten as $(a+b)^4 = a^2 + 6ab + b^3$ and then simplifying the equation will give us 'c'. 
    \item List given information: We only have one variable, so we need to find another.  
    $$=0 $$ which means that our unknown is $a+b$  or in other words  $(a-ba)^2 = c$ . This can be rewritten as a cubic equation and solved for 'c'. 
    \item Choose appropriate formula/method: We will use the method of completing squares.  
    $$=0 $$ which means that our unknown is $b$  or in other words  $(ab-bc)^2 = c$ . This can be rewritten as a quadratic equation and solved for 'c'. 
    \item Show calculations step by step: We start with the given information, then substitute it into one of the equations. In this case we have $a+b = 0$  so substituting that in gives us $$(ab-bc)^2 = c$$ which simplifies to  $abc - b^3 + a^4$ . Then simplify further and get our final answer as 'c'. 
    \item State Final Answer: The value of $a+b = 0$  is then the solution for variable "c".  
\end{enumerate}
The key concepts include exponentiation, addition/subtraction operations (which are fundamental to many mathematical problems), and solving equations using methods like completing squares.  This problem can be solved by a simple python code snippet as follows:   
```python    
import cv2   # Python's OpenCV library for image processing
from pytesseract import image_to_string, Output
# Load the tesseract-OCR engine and load your language data files  (english by default) using:   
pytesseract.pytesseract.tesseract_cmd = r'C:\Program Files\Tesseract-OCR\\tesseract.exe'   # path to tessearct executable file on your machine  (you may need adjust this)
# Load the image from a file or an URL:   
img = cv2.imread('question_image')     # replace 'question_image' with actual filename/URL of question picture   ('C:/Users... /Desktop/*file*.jpg', etc.)  (you may need adjust this)      img = ~cv2..imshow(0, imgsrc=img))
# Get the text from your image using pytesseract:   
text = image_to_string('question_image')   # replace 'question_image' with actual filename/URL of question picture  (you may need adjust this)      print(f"Question Text:\n{text}")     Output.LSTM64 is the output format, you can use other formats as well like:   
print('\n'.join([line for line in text.splitlines() if len(line)>10]))  # this will print only lines with more than one character   (you may need adjust these parameters to suit your needs). This is a very basic solution and might not work perfectly on all images, but it should be good enough as long you have the right image.
