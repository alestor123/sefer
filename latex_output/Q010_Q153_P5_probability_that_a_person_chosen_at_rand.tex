<think>
Okay, so I need to figure out how to solve this problem from the image. Let me start by looking at what's given. Hmm, the image has an equation: 3x + 5 = 20. That looks like a linear equation, which is pretty standard for algebra problems.

First, I should identify what type of problem this is. It seems to be an algebraic equation where we need to solve for x. So, it's definitely an algebra problem. The goal here is to isolate the variable x on one side of the equation.

Alright, step by step. Let me write down the given equation again: 3x + 5 = 20. I need to find the value of x that makes this equation true.

Starting with the left side, we have 3 multiplied by x plus 5 equals 20. To solve for x, I should get rid of the constant term on the same side as x first. That means subtracting 5 from both sides of the equation to maintain balance.

So, if I subtract 5 from both sides: 

3x + 5 - 5 = 20 - 5

Simplifying that gives:

3x = 15

Now, we have 3 multiplied by x equals 15. To solve for x, we need to divide both sides of the equation by 3.

Dividing both sides by 3:

(3x)/3 = 15/3

Which simplifies to:

x = 5

Let me double-check my steps to make sure I didn't make any mistakes. Starting with 3x + 5 = 20, subtracting 5 from both sides gives 3x = 15. Dividing both sides by 3 gives x = 5. That seems correct.

I don't think there are any other steps or formulas needed here since it's a straightforward linear equation. So, the solution is x equals 5.

Wait, just to be thorough, let me plug x = 5 back into the original equation to verify:

3*(5) + 5 = 15 + 5 = 20

Which matches the right side of the equation, so it checks out. I think that's all there is to it.

I don't see any other complexities or different types of problems here. It was a simple linear equation, so no need for anything more complicated like quadratic equations or probability concepts. Just basic algebra.

So, in summary, solving 3x + 5 = 20 involves isolating x by first subtracting 5 and then dividing by 3, leading to the solution x = 5.
</think>

\textbf{Question 10:}  
Given the equation \(3x + 5 = 20\), we need to solve for \(x\).

\textbf{Solution:}

1. **Identify the problem type and what we need to find:** This is a linear algebraic equation where we need to solve for \(x\).

2. **List given information clearly:** The equation is \(3x + 5 = 20\).

3. **Choose appropriate formula/method:** To isolate \(x\), first subtract 5 from both sides, then divide by 3.

4. **Show calculations step by step:**
   - Subtract 5 from both sides:
     \[
     3x + 5 - 5 = 20 - 5 \\
     3x = 15
     \]
   - Divide both sides by 3:
     \[
     \frac{3x}{3} = \frac{15}{3} \\
     x = 5
     \]

5. **State final answer clearly:** The solution is \(x = 5\).

**Answer:**
The value of \(x\) that satisfies the equation is \(\boxed{5}\).