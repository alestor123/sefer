Sure, I can help you understand the problem better by providing a solution for it using Python programming language in combination of numpy library which is used to handle mathematical operations on large multi-dimensional arrays efficiently (like matrices). 

\textbf{Question:} Suppose we have an array `arr` with values [10,23,-56] and another one called 'b' that contains the same data but in a different order. We want to find out if there is any common element between these two arrays using numpy library functions:
```python 
import numpy as np  
arr = np.array([10,23,-56])    # array 'a' with values [10,23,-56]    
b  = arr[::-1].copy()           # reversing the order of elements in b and copying it to a new variable ‘c’  
```                                                      
\textbf{Solution:}                                                                                           
Firstly we need to find common element between 'a' array (arr) & reverse sorted version ('b') using numpy library. We can use `np.intersect1d(x, y)[0]` function which returns the first intersection of two arrays: 
```python  
common_element = np.intersect1d(arr[::-1], b).tolist()    # getting common element from reversed 'b' and original order array (reversing it) using numpy library     [Note that the result is a list]     
print("Common Element: ",common_element ) 
```                                                                                                          
\textbf{Answer:} The Common elements in both arrays are -56.  
The key concepts used here include Numpy's array handling, slicing and reversal of an array using numpy library functions `np[::-1]` for the reverse order sort operation & function np.intersect() to find common element between two sorted lists/arrays in Python with NumPy (numpy).
